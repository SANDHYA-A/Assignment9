\documentclass[journal,12pt,twocolumn]{IEEEtran}

\usepackage{setspace}
\usepackage{gensymb}

\singlespacing


\usepackage[cmex10]{amsmath}
\usepackage{amsthm}

\usepackage{mathrsfs}
\usepackage{txfonts}
\usepackage{stfloats}
\usepackage{bm}
\usepackage{cite}
\usepackage{cases}
\usepackage{subfig}

\usepackage{longtable}
\usepackage{multirow}

\usepackage{enumitem}
\usepackage{mathtools}
\usepackage{steinmetz}
\usepackage{tikz}
\usepackage{circuitikz}
\usepackage{verbatim}
\usepackage{tfrupee}
\usepackage[breaklinks=true]{hyperref}
\usepackage{graphicx}
\usepackage{tkz-euclide}

\usetikzlibrary{calc,math}
\usepackage{listings}
    \usepackage{color}                                            
    \usepackage{array}                                            
    \usepackage{longtable}                                       
    \usepackage{calc}                                             
    \usepackage{multirow}                                         
    \usepackage{hhline}                                       
    \usepackage{ifthen}                                           
    \usepackage{lscape}     
\usepackage{multicol}
\usepackage{chngcntr}

\DeclareMathOperator*{\Res}{Res}

\renewcommand\thesection{\arabic{section}}
\renewcommand\thesubsection{\thesection.\arabic{subsection}}
\renewcommand\thesubsubsection{\thesubsection.\arabic{subsubsection}}

\renewcommand\thesectiondis{\arabic{section}}
\renewcommand\thesubsectiondis{\thesectiondis.\arabic{subsection}}
\renewcommand\thesubsubsectiondis{\thesubsectiondis.\arabic{subsubsection}}


\hyphenation{op-tical net-works semi-conduc-tor}
\def\inputGnumericTable{}                                 

\lstset{
%language=C,
frame=single, 
breaklines=true,
columns=fullflexible
}
\begin{document}


\newtheorem{theorem}{Theorem}[section]
\newtheorem{problem}{Problem}
\newtheorem{proposition}{Proposition}[section]
\newtheorem{lemma}{Lemma}[section]
\newtheorem{corollary}[theorem]{Corollary}
\newtheorem{example}{Example}[section]
\newtheorem{definition}[problem]{Definition}

\newcommand{\BEQA}{\begin{eqnarray}}
\newcommand{\EEQA}{\end{eqnarray}}
\newcommand{\define}{\stackrel{\triangle}{=}}
\bibliographystyle{IEEEtran}
\providecommand{\mbf}{\mathbf}
\providecommand{\pr}[1]{\ensuremath{\Pr\left(#1\right)}}
\providecommand{\qfunc}[1]{\ensuremath{Q\left(#1\right)}}
\providecommand{\sbrak}[1]{\ensuremath{{}\left[#1\right]}}
\providecommand{\lsbrak}[1]{\ensuremath{{}\left[#1\right.}}
\providecommand{\rsbrak}[1]{\ensuremath{{}\left.#1\right]}}
\providecommand{\brak}[1]{\ensuremath{\left(#1\right)}}
\providecommand{\lbrak}[1]{\ensuremath{\left(#1\right.}}
\providecommand{\rbrak}[1]{\ensuremath{\left.#1\right)}}
\providecommand{\cbrak}[1]{\ensuremath{\left\{#1\right\}}}
\providecommand{\lcbrak}[1]{\ensuremath{\left\{#1\right.}}
\providecommand{\rcbrak}[1]{\ensuremath{\left.#1\right\}}}
\theoremstyle{remark}
\newtheorem{rem}{Remark}
\newcommand{\sgn}{\mathop{\mathrm{sgn}}}
\providecommand{\abs}[1]{\left\vert#1\right\vert}
\providecommand{\res}[1]{\Res\displaylimits_{#1}} 
\providecommand{\norm}[1]{\left\lVert#1\right\rVert}
%\providecommand{\norm}[1]{\lVert#1\rVert}
\providecommand{\mtx}[1]{\mathbf{#1}}
\providecommand{\mean}[1]{E\left[ #1 \right]}
\providecommand{\fourier}{\overset{\mathcal{F}}{ \rightleftharpoons}}
%\providecommand{\hilbert}{\overset{\mathcal{H}}{ \rightleftharpoons}}
\providecommand{\system}{\overset{\mathcal{H}}{ \longleftrightarrow}}
	%\newcommand{\solution}[2]{\textbf{Solution:}{#1}}
\newcommand{\solution}{\noindent \textbf{Solution: }}
\newcommand{\cosec}{\,\text{cosec}\,}
\providecommand{\dec}[2]{\ensuremath{\overset{#1}{\underset{#2}{\gtrless}}}}
\newcommand{\myvec}[1]{\ensuremath{\begin{pmatrix}#1\end{pmatrix}}}
\newcommand{\mydet}[1]{\ensuremath{\begin{vmatrix}#1\end{vmatrix}}}
\numberwithin{equation}{subsection}
\makeatletter
\@addtoreset{figure}{problem}
\makeatother
\let\StandardTheFigure\thefigure
\let\vec\mathbf
\renewcommand{\thefigure}{\theproblem}
\def\putbox#1#2#3{\makebox[0in][l]{\makebox[#1][l]{}\raisebox{\baselineskip}[0in][0in]{\raisebox{#2}[0in][0in]{#3}}}}
     \def\rightbox#1{\makebox[0in][r]{#1}}
     \def\centbox#1{\makebox[0in]{#1}}
     \def\topbox#1{\raisebox{-\baselineskip}[0in][0in]{#1}}
     \def\midbox#1{\raisebox{-0.5\baselineskip}[0in][0in]{#1}}
\vspace{3cm}
\title{Matrix Theory EE5609\\
}

\author{\IEEEauthorblockN{Sandhya Addetla}\\
\IEEEauthorblockA{PhD Artificial Inteligence Department} \\
AI20RESCH14001\\
 }

\maketitle
\newpage
\bigskip
\renewcommand{\thefigure}{\theenumi}
\renewcommand{\thetable}{\theenumi}
\section*{Assignment-9}
\subsection*{Problem:}
Let $M_n$ denote the vector space of all $n\times n$ real matrices. Which of the following is a linear subspaces of $M_n$ :-
\begin{enumerate}
\item $ V_1 = \{  A \in M_n : \text{ A is nonsingular} \}$\\
\item $ V_2 = \{  A \in M_n : det(A) = 0 \}$\\
\item $ V_3 = \{  A \in M_n : trace(A) = 0 \}$\\
\item $ V_4 = \{  BA : A \in M_n\},$ where $ B$ is some fixed matrix in $ M_n$
\end{enumerate}
\subsection*{Solution:}
\begin{table}[h!]
\begin{center}
\begin{tabular}{ | m{3cm} | m{5cm}| } \hline 
%\begin{tabular}{|c|c|}\hline
Vector space & Is it subspace to $M_n$?\\  \hline
1)$V_1$: All non-singular matrices of $n \times n$ &   The matrices $I_{n\times n}$ and $-I_{n\times n}$ are non-singular matrices, but the sum $I_{n\times n}-I_{n\times n}$ is zero matrix and it is singular. \\& $\therefore V_1$ does not form subspace of $ M_n$.\\
\hline
2)$V_2$: All singular matrices of $n \times n$ &    The matrices $\myvec{1&0\\0&0} and \myvec{0 &0\\0&1}$  are singular matrices, but the sum is a non-singular matrix. \\& $\therefore V_2$ does not form subspace  $ M_n$.\\
\hline
3)$V_3$: All matrices of $n \times n$ with trace =0&    Let $ \vec{v_1}$ and $ \vec{v_2}$ be matrices with Trace = 0. \\& $Tr(\vec{v_1} + \alpha\vec{v_2}) = Tr(\vec{v_1}) + \alpha Tr(\vec{v_2}) = 0$. \\& $\therefore$ the vector space $V_3$  forms linear subspace of  $ M_n$.\\\hline
4)$V_4$: $F_A$ = BA, where B  is some fixed matrix in $ M_n$&     Let $ \vec{v_1}$ and $ \vec{v_2}$ be matrices in the vector space $V_4$.\\& $F_{v_1 +\alpha v_2}$ = $B(\vec{v_1} + \alpha\vec{v_2})$\\& =$ B\vec{v_1} + \alpha B\vec{v_2}  =$\\& $ F_{v_1} + \alpha F_{v_2} $.\\& $\therefore V_4$   forms linear subspace of $ M_n$.\\
\hline
\end{tabular}
\end{center}
\end{table}
\end{document}